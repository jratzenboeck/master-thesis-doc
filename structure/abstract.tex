\chapter*{Abstract}
\markright{Abstract}

Customer satisfaction is a key criteria for modern businesses as it is an important ingredient for establishing a long-term relationship with a customer. Unfortunately, conducting a survey is only a temporary solution as the satisfaction level is very volatile. 

This thesis tried to tackle the existing issue by researching a methodology with which it should be possible to reason about customer satisfaction automatically using provided data from a large software system. Based on existing literature about customer satisfaction models, the researcher investigated relevant data sources with supposed influence on the satisfaction outcome of customers. The work presents the cornerstones of the implemented data extraction tool to bring data into the desired format for analysis. It then outlines the two different approaches applied on the extracted data to derive desired knowledge about influential factors and predict whether a customer tends to be (dis)satisfied. 

The first approach starts based on predefined business critical hypotheses with regard to customer satisfaction in order to find relationships within the data and thus filter satisfaction critical attributes. The thesis exemplarily illustrates the analysis with certain statistical tests for verifying respectively falsifying the hypotheses. 

The second approach presents a knowledge discovery process applied on customer feedback data gathered from a satisfaction survey. A framework implementing researched data mining algorithms will be outlined step-wise. 

Finally, it could be shown that an optimally configured Random forest classifier is able to classify about 75\% of customers correctly as either dissatisfied or satisfied. Hence, it clearly outperformed the support vector machine classifier. The evaluation emphasizes that customer satisfaction is not yet well represented in collected data and thus the research reveals room for improvement. 

