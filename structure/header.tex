\documentclass[11pt,a4paper,titlepage,
chapterprefix,headsepline,parskip,pdftex,
,pointlessnumbers,bibtotoc]{scrreprt}

%%% paragraph for lower depth
\makeatletter %% remove @ meaning
\renewcommand{\paragraph}{\@startsection
   {paragraph} % name
   {4} % depth
   {0mm} % indention
   {-\baselineskip} % before
   {0.1\baselineskip} % after
   {\normalfont\normalsize\bfseries}} % stil
\makeatother %% add @ meaning

\makeatletter %% remove @ meaning
\renewcommand{\subparagraph}{\@startsection
   {subparagraph} % name
   {5} % depth
   {0mm} % indention
   {-\baselineskip} % before
   {0.1\baselineskip} % after
   {\normalfont\normalsize\bfseries}} % stil
\makeatother %% add @ meaning

\usepackage{setspace}
\onehalfspacing

\usepackage[pdftex]{graphicx}

% for colours
\usepackage[pdftex]{color}

% Configure defintions
\usepackage{amsthm}
\newtheorem{mydef}{Definition}

\usepackage{tikz}
\usetikzlibrary{shapes,arrows}
% \usepackage{subfigure} // cannot be used with subcaption
\usepackage{caption}
\usepackage{subcaption}

\usepackage[colorlinks=true,
    linkcolor=black,
    citecolor=black,
    pagecolor=black,
    urlcolor=black,
    breaklinks=true,
    bookmarksnumbered=true,
    hypertexnames=false,
    pdfpagemode=UseOutlines,
    pdfview=FitH,
    plainpages=false,
    pdfpagelabels,
    bookmarks=true,
    linktocpage=true]{hyperref}

\hypersetup{pdfauthor={J�rgen Ratzenb�ck},
    pdftitle={Web Performance Monitoring and Tuning},
    pdfsubject={Bachelor's Thesis},
    pdfkeywords={},
    pdfcreator={pdfLaTeX with hyperref (\today})}

%%% Source-Code
\usepackage{listings}

\lstset{% general command to set parameter(s)
basicstyle=\small, % print whole listing small
tabsize=2, %
keywordstyle=\color[rgb]{0.00,0.00,0.50}{}\bfseries,
%identifierstyle=, % nothing happens
commentstyle=\color[rgb]{0.00,0.50,0.25}{},
%stringstyle=\ttfamily, % typewriter type for strings
%showstringspaces=false} % no special string spaces
numbers=left, numberstyle=\tiny, numbersep=5pt}

\colorlet{punct}{red!60!black}
\definecolor{background}{HTML}{EEEEEE}
\definecolor{delim}{RGB}{20,105,176}
\colorlet{numb}{magenta!60!black}

% Custom definition for JSON
\lstdefinelanguage{json}{
    basicstyle=\normalfont\ttfamily,
    numbers=left,
    numberstyle=\scriptsize,
    stepnumber=1,
    numbersep=8pt,
    showstringspaces=false,
    breaklines=true,
    frame=lines,
    backgroundcolor=\color{background},
    literate=
     *{0}{{{\color{numb}0}}}{1}
      {1}{{{\color{numb}1}}}{1}
      {2}{{{\color{numb}2}}}{1}
      {3}{{{\color{numb}3}}}{1}
      {4}{{{\color{numb}4}}}{1}
      {5}{{{\color{numb}5}}}{1}
      {6}{{{\color{numb}6}}}{1}
      {7}{{{\color{numb}7}}}{1}
      {8}{{{\color{numb}8}}}{1}
      {9}{{{\color{numb}9}}}{1}
      {:}{{{\color{punct}{:}}}}{1}
      {,}{{{\color{punct}{,}}}}{1}
      {\{}{{{\color{delim}{\{}}}}{1}
      {\}}{{{\color{delim}{\}}}}}{1}
      {[}{{{\color{delim}{[}}}}{1}
      {]}{{{\color{delim}{]}}}}{1},
}

\lstset{
    basicstyle=\normalfont\ttfamily,
    numbers=left,
    numberstyle=\scriptsize,
    stepnumber=1,
    numbersep=8pt,
    showstringspaces=false,
    breaklines=true,
    frame=lines,
    backgroundcolor=\color{background},
}

% Javascript formatting
\lstdefinelanguage{JavaScript}{
  keywords={typeof, new, true, false, catch, function, return, null, catch, switch, var, if, in, while, do, else, case, break},
  keywordstyle=\color{blue}\bfseries,
  ndkeywords={class, export, boolean, throw, implements, import, this},
  ndkeywordstyle=\color{darkgray}\bfseries,
  identifierstyle=\color{black},
  sensitive=false,
  comment=[l]{//},
  morecomment=[s]{/*}{*/},
  commentstyle=\color{purple}\ttfamily,
  stringstyle=\color{red}\ttfamily,
  morestring=[b]',
  morestring=[b]"
}

\lstset{
   language=JavaScript,
   backgroundcolor=\color{lightgray},
   extendedchars=true,
   basicstyle=\footnotesize\ttfamily,
   showstringspaces=false,
   showspaces=false,
   numbers=left,
   numberstyle=\footnotesize,
   numbersep=9pt,
   tabsize=2,
   breaklines=true,
   showtabs=false,
   captionpos=b
}


% Figure
\newcommand{\cffigure}[1]{\hyperref[#1]
{cf. \ref*{#1}~\nameref{#1}}}
\usepackage{float}

%%% continous footnote
\newcounter{cfootnotecounter}
\newcommand{\cfootnote}[1]{\stepcounter{cfootnotecounter}
\footnote[\value{cfootnotecounter}]{#1}}

\flushbottom

% change page settings
\setlength{\hoffset}{0mm} \setlength{\voffset}{0mm}
\setlength{\evensidemargin}{14.6mm}
\setlength{\oddsidemargin}{14.6mm} \setlength{\topmargin}{-20mm}
\setlength{\headheight}{15mm} \setlength{\headsep}{9mm}
\setlength{\textheight}{242mm} \setlength{\textwidth}{145mm}
\setlength{\footskip}{10mm}


%%% seperation of float-environment
\setlength{\textfloatsep}{25pt plus5pt minus5pt}
\setlength{\intextsep}{25pt plus5pt minus5pt}

%%% Gliederungs-Nummern in den Rand schreiben
\renewcommand*{\othersectionlevelsformat}[1]{%
\llap{\csname the#1\endcsname\autodot\enskip}}

%%% only chapter name in headline
\renewcommand*{\chaptermarkformat}{}

%%% format of the chapter looks
\setkomafont{chapter}{\Huge}
\renewcommand*{\chapterformat}{\LARGE{\chapappifchapterprefix{\ }\thechapter\autodot\enskip}}

%%% headline
\usepackage[automark]{scrpage2}

\clearscrheadings \clearscrplain \clearscrheadfoot
\pagestyle{scrheadings}
\ohead{\pagemark}
\ihead{\headmark}
\cfoot{}

%%% design of chapter pages
\renewcommand*{\chapterpagestyle}{scrheadings}

%% indices from TOC and numbering depth
\setcounter{tocdepth}{\subsubsectionlevel}
\setcounter{secnumdepth}{\paragraphlevel}

%%% Array for tables
\usepackage{array}

%%% fonts
\addtokomafont{chapter}{\sffamily}
\addtokomafont{sectioning}{\rmfamily}

% language
\usepackage[english]{babel}
% inputencoding
\usepackage[utf8]{inputenc}
% fontencoding
\usepackage[T1]{fontenc}
\usepackage{ae}

% URLs
\usepackage{url}

%%% orphan  und widow
\clubpenalty = 10000
\widowpenalty = 10000 \displaywidowpenalty = 10000


%%% Einbinden von kompletten PDF-Seiten
\usepackage{pdfpages}
