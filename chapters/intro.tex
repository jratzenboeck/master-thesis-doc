\chapter{Introduction}
\label{ch:intro}
This first chapter sets the context of this thesis by outlining a shift in business strategies over the last decades pointing out the importance of customers as key figure in this development. Based on these findings the actual problem to be solved as well as the research objectives will be defined. 

\section{Problem statement}
\label{sec:problem}
Until the 1980s companies were only focused on the quality of their products and services. Demand at this time was usually high and, due to little competition on the market, customers discovered the product they wanted to have more easily. Companies focused on optimizing product quality based on internal standards. An important task was to collect and monitor internal data about quality developments in the product life cycle to assess the improvement of the internal processes over time. However, increasing supply and resulting competition on the market caused companies to shift towards a more customer centric approach starting in the early 1990s. As online services and products have become more popular and successful over the next few years, the dramatic growth of oversupply and market intensity required companies to start building their whole business around customer needs and wants to stay profitable in the long-term. \cite{neckel2015}. It was for instance shown that an 5\% increase of customer retention has a positive impact of 25 to 125\% on the profit \cite{bowen2001relationship}.

In such a competitive market, customers rarely stay at the same service provider due to convenience only. Switching to a competitor has never been easier as it is in most parts of the software industry \cite{rygielski2002data}. However, for businesses it gets harder to attract and acquire new customers which led companies to treat a purchase of a new customer as the start of a long-term relationship. Keeping customers over longer periods turns out to be a major business issue nowadays \cite{nerdinger2015} \cite{neckel2015}. Attracting and acquiring a new customer instead requires much more costs than selling to an existing loyal customer \cite{aydin2006switching}. There has been a lot of research going on for many years with the goal to reason about the main drivers and their effect on customer retention, whereby customer satisfaction has been identified as a central factor as proved by large-scale analysis in the industry \cite{fornell1992national} \cite{bolton1998dynamic} \cite{gustafsson2005effects}. Although there has been an ongoing debate on the strength of this effect, software businesses are subject of a heteroegenous industry and thus more affected by the satisfaction of their customers in contrast to homegeneous industries \cite{fornell1992national}. As a result of these findings, competitive software businesses have to find a way to treat their customers carefully and ensure their happiness over a longer period. 

Fortunately technology in the software industry has evolved dramatically. The Web 2.0 capabilities allow to collect huge amounts of customer data while computational power of hardware enables businesses to store every single touch point of customers in large-scale database systems and analyze them efficiently \cite{chen2003understanding} \cite{neckel2015}. Data mining techniques allow to derive patterns based on customer data turned into comprehensive knowledge and moreover provide opportunities to predict future events depending on them. Despite the massive ongoing research in the field of data anlaysis and mining related to customer knowledge, automatic reasoning about customer satisfaction with regard to an Internet business lacks major research success. Although businesses nowadays leverage a huge amount of customer data, invest a lot of resources to treat their customers as personal as possible and usually employ a first-class customer support service to quickly tackle problems and complaints, this thesis sees room for improvement in assessing satisfaction of any customers pro-actively and automatically \cite{neckel2015}.

\section{Research objectives}
The research question of this thesis is about finding out how to leverage useful data related to the behavior of customers and their interaction with services to derive patterns which allow to make a statement on how (dis)satisfied a customer is. Furthermore the results should show which data have major influence on customers satisfactory level and thus outline the essential product characteristics a service provider has to pay attention to and optimize to be successful in the future. If the results can be considered as reliable and valid, they can be used for early detection of unsatisfied customers and help to pro-actively solve their problems instead of handling their problem in a reactive manner.
The route towards the goal of the research looks the following:

\begin{itemize}
	\item Define what customer satisfaction means and how to differentiate between satisfactory levels.
	\item Elaborate which data has potential to influence customer satisfaction regarding the case study example used in this research. 
	\item Find a way to represent customer satisfaction as reliable and measurable metric.
	\item Implement statistical tests to find out which metrics in the data affect customer satisfaction. 
	\item Implement a software solution to analyze behavior of customers, learn from this data and do a predictive analysis of how (dis)satisfied an arbitrary customer will be.
\end{itemize}

\section{Case Study: Tractive}
\label{sec:illustrationExample}
As illustrative example the research relies on the data of a pet tracking company named Tractive. The company which was founded as a startup in 2012 in Pasching, Upper Austria produces hardware devices and software applications for different kinds of pets. Their most popular product is a GPS (Global Positioning System) device which can be put on a pet and allows the customer to track its position live on a mobile phone or via a website. This way Tractive helps pet owners worldwide not to lose their pets again since they can always have an eye on them via a smartphone or desktop computer. Since the company is quite successful with more than 100k customers using a Tractive GPS device, there is also a lot of data available related to each customer. This starts with data related to the usage of the hardware device ranging to subscription data of a customer and his/her interaction with customer support. Since the company's business model is based on subscriptions which require customers to pay on a regular basis for the usage of the device, there is special interest in binding customers for a long time to the company. In essence, this means that satisfied customers are a key asset for the company. Therefore its model should fit well to the research objectives and promising results could support solving customer issues earlier and as a result affect drop out rate positively. 

\section{Structure of the Thesis}
The thesis will contribute to a more efficient way of measuring customer satisfaction using large sets of collected data related to usage behavior. Chapter~\ref{ch:intro} first gives an introduction about the general problem context and outlines the research objectives as well as the case study this thesis is based on. Chapter~\ref{ch:backgroundResearch} will take a closer look at related research work. It will define the term customer satisfaction and its limitations. It gives an overview on existing approaches to measure customer satisfaction and discusses their efficiency, problems and improvement potential. Furthermore, this Chapter outlines necessary background research done to gain a deeper understanding on what it means to analyze satisfaction and sheds some light onto different methods suitable for the implementation part. Following, chapter~\ref{ch:implementation} outlines the process of selecting relevant data sources to be used for customer satisfaction determination. Subsequently, the chapter illustrates statistical analysis methods applied on this data to reason about relationships of particular data collections and attributes. Chapter \ref{ch:dataDriven} illustrates an approach of learning from explicit customer feedback in order to predict satisfaction of arbitrary customers based on collected data about them. Chapter~\ref{ch:evaluation} evaluates the outcome of this implementation and compares the approaches regarding their performance results retrieved. Finally, Chapter~\ref{ch:conclusion} concludes the work by summarizing the important findings and  a short look into future work.