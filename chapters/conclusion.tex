\chapter{Conclusion and Future Work}
\label{ch:conclusion}

The work in this thesis conducted methods of analyzing customer behavior, interaction data and product quality characteristics to detect a customer's satisfaction respectively dissatisfaction automatically and on-demand. The contribution of this work should help businesses, processing large amounts of data, to detect potentially dissatisfied customers at an early stage to tackle their issues or concerns pro-actively and as a result prevent them from churning or switching to a competitor. From a research perspective, the work sheds light on the complexity of the customer satisfaction construct and showed how collected data can be leveraged with scientific approaches to turn gathered knowledge about customers into a judgment of their satisfaction. Particular data insights and results of the work, revealed different questions and problems to be solved in future research work whereby some of them will be mentioned at the end of this chapter. The following paragraphs briefly summarize the cornerstones of this thesis and draw essential conclusions.

In the first part, the objective was to gather enough knowledge on the topic of customer satisfaction itself. Based on the original definition of comparing a customer's individual expectations with the perceived value of a product, different satisfaction models were analyzed and matched regarding its applicability with the concept of this thesis' case study, a pet tracking company with more than 100k customers. Based on a selected satisfaction model, an intensive analysis of data collections within the available database system of the case study was conducted to filter data sources considered as influencing factor for customer satisfaction. The implementation of an extraction tool brought the data in an appropriate form for analyzing it with regard to customer satisfaction. Two approaches with different philosophies were considered to have a detailed look on the data and find desired patterns. Firstly, the author started with a top-down analysis where the goal was to identify essential features driving customer satisfaction metrics based on business relevant hypotheses defined prior to the individual analysis task. Implicit data as the mobile app usage or the frequency of device usage were considered as customer satisfaction predictor. Using statistical tests those hypotheses were verified respectively falsified. The thesis showed exemplarily how those statistical tests were applied. Due to the weak correlations in the data, it was necessary to gather explicit satisfaction feedback from customers to start with a data-driven knowledge discovery process. The extracted feedback from the conducted satisfaction survey was merged with the relevant data sources collected. With the knowledge of mature data mining research tools a framework was implemented which automatically preprocessed input data accordingly to overcome weaknesses in data quality, conducted feature selection techniques in addition to the manual feature construction done before and learned a model based on machine learning techniques, like decision trees and support vector machines, which was then used for classifying customers by their satisfaction rating. 

The final results retrieved from the data-driven approach, revealed at first glance a remarkable difference in classification power between the support vector machine and the Random forest approach. Due to the limitations of this master thesis topic, no detailed insights into the reasons for the bad performance of SVM respectively the better suitability of the data set for a rule-based approach, were taken. The effort put into various preprocessing approaches but especially into the correct handling of the class imbalance problem were shown to be of major importance in a knowledge discovery process. Both, the statistical and data-driven analysis manifested the difficulty of finding evidence for customer satisfaction in collected behavioral data of a customer, at least in the given data by the case study. The achieved classification presented in the previous chapter shows that the implemented framework can provide a tendency about a customer's attitude towards his or her satisfaction level but based on the current state there is a lot of room for improvement. The almost completely missing statistically significant correlations in the data emphasize the difficulty of revealing relevant attributes which contribute to the customer satisfaction model. 

Regarding the future, binding customers over a long term will continue to be essential for profitability of a business. Therefore, constantly monitoring customer satisfaction would be of great interest and help for businesses. Future research work has to put special emphasis on the fact of matching correct raw data onto satisfaction metrics. In addition to the focus on quality features, including emotional data of customers could lead to additional expressive features. Possible data could be for instance fetched via sentiment analysis from customer reviews or communication with the help center of a business. Research successes in natural language processing can contribute positively in the future when being able to reliably extract emotional feelings of a customer based on his or her utterances. With regard to the technical development, other feature engineering techniques like PCA (Principal Component Analysis) or automatic feature learning methods can be applied to improve the feature engineering process. In any way, the research area of data mining is still very busy and could bring further interesting analysis tools in the near future. 